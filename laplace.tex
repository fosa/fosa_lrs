%coding:utf-8

%----------------------------------------
%FOSAPHY, a LaTeX-Code for a summary of linear systems and regulation
%Copyright (C) 2014, Mario Felder, Michael Fallegger

%This program is free software; you can redistribute it and/or
%modify it under the terms of the GNU General Public License
%as published by the Free Software Foundation; either version 2
%of the License, or (at your option) any later version.

%This program is distributed in the hope that it will be useful,
%but WITHOUT ANY WARRANTY; without even the implied warranty of
%MERCHANTABILITY or FITNESS FOR A PARTICULAR PURPOSE.  See the
%GNU General Public License for more details.
%----------------------------------------

\chapter{Laplace Transformation}
\section{Definition}
Definition der Laplace-Transformierten $U(s)$ eines Signals $u(t)$:
\[
	 \lap\{u(t)\} = \int_{0^-}^{\infty} u(t) \cdot \e^{-st} \di t = \lim\limits_{a \nearrow 0} \int_{a}^{\infty}  u(t) \cdot \e^{-st} \di t
\]
\\
Notation:
\[
	u(t) \ \laplace \ U(s) \qquad ,s \in \mathbb{C}
\]

\subsection{Konvergenzbereich}
Die Laplace-Transformierte eines Signals existiert nicht für jedes $s$.
Falls das Integral
\[
	U(s) = \int_{0}^{\infty} u(t) \cdot \e^{-st} \di t
\]
konvergiert, so existiert die Laplace-Transformierte von $u(t)$.\\
Der Bereich aller Zahlen $s$, für welche die Laplace-Transformierte eines Signals konvergiert, den Konvergenzbereich (KB).
\[
	s = \sigma + \omega \cdot \im \qquad , \sigma = Re(s) \text{ und } \omega = Im(s)
\]
Somit ist der Konvergenzbereich:
\[
	KB = \{ s \in \mathbb{C} | Re(s) > \sigma_0 \}
\]
\\\\


\section{Eigenschaften der Laplace-Transformation}
\paragraph{Linearitätssatz}
\[
	A \cdot u(t) + B \cdot v(t) \ \laplace\ A \cdot U(s) + B \cdot V(s) 
\]

\paragraph{Ähnlichkeitssatz}
\[
	u(a \cdot t)\ \laplace \ \frac{1}{a} \cdot U\left(\frac{s}{a}\right) \qquad ,a>0
\]

\paragraph{Dämpfungssatz}
\[
	\e^{-a\cdot t}u(t) \ \laplace \ U(s + a) \qquad ,a>0	
\]

\paragraph{Zeitverschiebungssatz}
\[
	t(t-t_0) \cdot \sigma (t-t_0) \ \laplace \ \e^{-t_0 \cdot s} \cdot U(s)	\qquad ,t_0 \geq 0
\]

\paragraph{Faltungssatz}
\[
	u(t) \ast v(t) \ \laplace \ U(s) \cdot V(s)
\]

\paragraph{Differentiationssatz}
\[\begin{aligned}
	\dot{u}(t) \ &\laplace \ s \cdot U(s) - u(0^-) \\
	\ddot{u}(t) \ &\laplace \ s^2 \cdot U(s) - s \cdot u(0^-) -  \dot{u}(0^-) \\
	\dddot{u}(t) \ &\laplace \ s^3 \cdot U(s) - s^2 \cdot u(0^-) - s \cdot \dot{u}(0^-) - \ddot{u}(0^-)\\
	\\
	u^{(n)}(t) \ &\laplace \ s^n \cdot U(s) - s^{n-1} \cdot u(0^-) - s^{n-2} \cdot \dot{u}(0^-) - \dots - s \cdot u^{n-2}(0^-) - u^n(0^-)
\end{aligned}\]
~\\
Mit Anfangsbedingung $0^- = 0$:
\[
	\difrac{^n}{t^n}u(t) \ \laplace \ s^n \cdot U(s)
\]

\paragraph{Integrationssatz}
\[
	\int_{0^-}^{t} \di t \ \laplace \ \frac{1}{s}\cdot U(s)
\]
\\

\paragraph{Inverse Laplace-Transformation}
\[
	\lap^{-1}\{U(s)\} = u(t)
\]
\[
	U(s) \ \Laplace \ u(t)
\]
\\Eindeutigkeitssatz:
\[
	U(s) = V(s) \ \Laplace \ u(t) = v(t) \qquad , \text{für }t>0
\]
\\\\

\section{Partialbruchzerlegung}
Rationale Funktion:
\[
	R(s) = \frac{Z(s)}{N(s)}
\]
\\
Wobei $Z(s)$, und $N(s)$ Polynome in $s$ sind. 
\\
$Z(s) > N(s)$ unecht gebrochen, $Z(s) < N(s)$ gebrochen\\
\\
In Linearfaktoren zerlegen:
\[
	R(s) = \frac{Z(s)}{(s-s_1)(s-s_2)\ldots(s-s_m)}
\]
Wobei $s_1,\ldots ,s_m$ komplexe Polstellen von $R(s)$ sind.

\subsection{Rationale Funktionen mit einfache Polen}
\[
	R(s) = \frac{Z(s)}{(s-s_1)(s-s_2)\ldots(s-s_m)} \qquad , s_1,\ldots ,s_m \text{ paarweise verschieden}
\]
\\
Es gibt komplexe Zahlen $a_1,\ldots ,a_m$, so dass
\[
	R(s)=\frac{a_1}{s-s_1}+\frac{a_2}{s-s_2}+\cdots+\frac{a_m}{s-s_m}
\]
\\
Bestimmung der komplexen Zahlen $a_i$:
\[
	a_i=\left.R(s)\cdot(s-s_1)\right|_{s=s_i}
\]
\\

\subsection{Rationale Funktionen mit mehrfachen Polen}
\[
	R(s) = \frac{Z(s)}{(s-a)^,} \qquad a \in \mathbb{C}
\]
\\
Es gibt komplexe Zahlen $a_1,\ldots ,a_m$, so dass
\[
	R(s)=\frac{a_1}{s-a}+\frac{a_2}{(s-a)^1}+\cdots+\frac{a_m}{(s-a)^m}
\]
\\
Bestimmung der komplexen Zahlen $a_i$:
\[
	a_{m-i} = \frac{1}{i!}\left[\difrac{^i}{s^i}R(s)(s-a)^a\right]_{s=a} \qquad i=0,\ldots,m-1
\]
\\\\

\section{Lösen von Differentialgleichungen}
\begin{enumerate}
	\item Differentialgleichung in Bildbereich überführen $\rightarrow$ algebraische Gleichung
	\item algebraische Gleichung im Bildraum nach der unbekannten Funktion auflösen
	\item Bildfunktion des Eingangssignals bestimmen und in die algebraische Gleichung einsetzen
	\item Lösung aus dem Bildraum in den Zeitraum zurück transformieren
\end{enumerate}

\section{Übertragungsgleichung LZI-Systemen}
Der Zusammenhang zwischen Eingangssignal $u(t)$ und Ausgangssignal $v(t)$ ist durch eine lineare Differentialgleichung mit konstanten Koeffizienten beschrieben
\[
	a_n\cdot v^{(n)} + a_{n-1} \cdot v^{(n-1)} + \ldots + a_1 \cdot \dot{v} +a_0\cdot v = b_m\cdot u^{(n)} + b_{n-1} \cdot u^{(n-1)} + \ldots + b_0 \cdot u 
\]
\\
Da nur kausale Signale betrachtet werden, entfallen die Anfangsbedingungen:
\[
	a_ns^nV+a_{n-1}s^{n-1}V+\ldots+a_1sV+a_0V=b_ms^mU+b_{m-1}s^{m-1}U+\ldots+b_0U
\]
Nach $V(s)$ aufgelöst ergibt dies:
\[
	V(s)=\underbrace{\frac{b_ms^m + b_{m-1}s^{m-1}+\ldots+b_1s + b_0}{a_ns^n + a_{n-1}s^{n-1}+\ldots+a_1s + a_0}}_{G(s)} \cdot U(s)
\]
\\
Die rationale Funktion $G(s)$ wird Übertragungsfunktion des Systems genannt. Daraus ergibt sich die Übertragungsgleichung:
\[
	V(s) = G(s) \cdot U(s)
\]
\\\\

\section{Faltung}
Die Faltung ist definiert durch:
\[
	(f \ast g)(t) = \int_{-\infty}^{\infty}f(\tau) \cdot g(t-\tau) \di \tau
\]
\\
Die Faltung in den Bildbereich transformiert:
\[
	(f \ast g)(t) \ \laplace \ F(s) \cdot G(s)
\]

\paragraph{Rechenregeln}
\begin{enumerate}
	\item Dirac-Delta ist das neutrale Element bzgl. der Faltung: $\delta(t-t_0) \ast g(t) = g(t-t_0)$
	\item Kommutativität: $f\ast g = g \ast f$
	\item Assoziativität: $f \ast (g \ast h) = (f \ast g) \ast h$
	\item Distributivität: $f \ast (g + h) = f \ast g + f \ast h$
\end{enumerate}

\subsection{Gewichtsfunktion}
Aus dem Faltungssatz ergibt sich:
\[
	v(t) = g(t) \ast u(t) \qquad , g(t) \ \laplace \ G(s)
\]
\\
$g(t)$ wird Gewichtsfunktion genannt.
\\

\subsection{Impulsantwort}
Die Impulsantwort ist definiert als die Antwort auf einen Einheitsimpuls $u(t) = \delta(t)$ zur Zeit $t=0$.
\[
	v(t) = g(t) \ast u(t) = g(t) \ast \delta(t) = g(t)
\]
\\
Die Impulsantwort ist die Gewichtsfunktion.
\\

\subsection{Sprungantwort}
Als Sprungantwort $h(t)$ bezeichnet man die Antwort des Systems auf einen Einheitssprung $\sigma(t)$:
\[
	h(t) = g(t) \ast \sigma(t) = \lap^{-1}\left\lbrace G(s) \cdot \frac{1}{s} \right\rbrace = \int_{0^-}^{t} g(\tau) \di \tau
\]
\\

\subsection{Anfangswertsatz}
In gewissen Situationen ist das Verhalten des Systems kurz nach dem Einschalten interessant:
\[
	v(0^+) = \lim\limits_{t \searrow 0}v(t)
\]
\\
Dies kann direkt im Bildraum berechnet werden. Ist $v(t)$ ein in $t=0$ sprungstetiges Signal, so existiert der Anfangswert und es gilt:
\[
	v(0^+) = \lim\limits_{Re(s)\rightarrow + \infty} s\cdot V(s)
\]

\subsection{Endwertsatz}
Ist der Endwert $v(\infty)$ eines Signals $v(t)$ gefragt:
\[
	v(\infty) = \lim\limits_{t \rightarrow \infty} v(t)
\]
, gilt der folgende Satz:\\
Ist $v(t)$ ein Signal, für welches der Endwert $v(\infty)$ existiert, so gilt:
\[
	v(\infty) = \lim\limits_{s \rightarrow 0} s \cdot V(s)
\]
\\
Der Endwert existiert, wenn alle Pole der Bildfunktion $V(s)$ links der imaginären Achse ($Re(s)<0$) sind. Ausnahme ist eine einfache Polstelle bei $s=0$.
\\

\subsection{Stabilität}
Ein System ist symptotisch stabil, falls seine Impulsantwort $g(t)$ mit $t \rightarrow \infty$ gegen Null abklingt:
\[
	g(\infty) = 0
\]
\\
Diese Eigenschaft ist äquivalent dazu, dass alle Polstellen der Übertragungsfunktion $G(s)$ links der imaginären Achse liegen.