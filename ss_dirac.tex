%coding:utf-8

%----------------------------------------
%FOSAPHY, a LaTeX-Code for a summary of linear systems and regulation
%Copyright (C) 2014, Mario Felder, Michael Fallegger

%This program is free software; you can redistribute it and/or
%modify it under the terms of the GNU General Public License
%as published by the Free Software Foundation; either version 2
%of the License, or (at your option) any later version.

%This program is distributed in the hope that it will be useful,
%but WITHOUT ANY WARRANTY; without even the implied warranty of
%MERCHANTABILITY or FITNESS FOR A PARTICULAR PURPOSE.  See the
%GNU General Public License for more details.
%----------------------------------------

\section{Dirac-Delta-Funktion}
Die Dirca-Delta-Funktion ist definiert als Ableitung des Einheitssprungs:
\[
	\delta(t) = \difrac{\sigma}{t}
\]
\\
Sie hat Punktweise folgende Werte:
\[
	\delta(t) = \left\lbrace \begin{matrix}
		0 \qquad \text{für }t \neq 0\\
		\infty \qquad \text{für }t = 0.
	\end{matrix} \right.
\]
\\
Es sollte jedoch nur unter dem Integral gerechnet werden:
\[
	\int_{-\infty}^{\infty} \delta(t) \di t = 1
\]

\subsection{Ausblendefunktion}
Ist die Funktion $u(t)$ an der Stelle $t_0$ stetig, so gilt:
\[
	\int_{-\infty}^{\infty} u(t) \cdot \delta(t-t_0) \di t = u(t_0)
\]
oder:
\[
	u(t) \cdot \delta(t-t_0) = u(t_0) \cdot \delta(t-t_0)
\]

\subsection{Verallgemeinerte Ableitung}
Es ist definiert:
\[
	\difrac{}{t}(A \cdot \sigma(t-t_0)) := A \cdot \delta(t-t_0)
\]