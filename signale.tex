%coding:utf-8

%----------------------------------------
%FOSAPHY, a LaTeX-Code for a summary of linear systems and regulation
%Copyright (C) 2014, Mario Felder, Michael Fallegger

%This program is free software; you can redistribute it and/or
%modify it under the terms of the GNU General Public License
%as published by the Free Software Foundation; either version 2
%of the License, or (at your option) any later version.

%This program is distributed in the hope that it will be useful,
%but WITHOUT ANY WARRANTY; without even the implied warranty of
%MERCHANTABILITY or FITNESS FOR A PARTICULAR PURPOSE.  See the
%GNU General Public License for more details.
%----------------------------------------

\chapter{Systeme und Signale}
\section{Signale}
\subsection{Definition}
Ein Signal ist eine (reelle) Funktion:
\[
	u:\ \mathbb{R} \rightarrow \mathbb{R}
\]


\subsection{Einheitssprung}
Der Einheitssprung wird in der Technik oft gebraucht und ist folgendermassen definiert:
\[
	\sigma := \left\lbrace 
		\begin{matrix}
			1 \qquad \text{für alle } t \ge 0.\\
			0 \qquad \text{für alle } t < 0.
		\end{matrix} \right.
\]
\\
Eine weitere Bezeichnung lautet $H(t)$, Heaviside-Funktion.


\subsection{Eigenschaften}
\paragraph{Sprungstelle:}
Ist eine Funktion $u(t)$ in einem Punkt $t_0$ definiert aber unstetig, so heisst $t_0$ eine Sprungstelle von $u(t)$.\\
Wenn die einseitigen Grenzwerte $\lim\limits_{t \nearrow t_0} u(t)$ und $\lim\limits_{t \searrow t_0} u(t)$ existieren und endlich sind, so heisst die Sprungstelle endlich.
\paragraph{Knickstelle:}
Ist $u(t)$ in $t_0$ stetig, aber nicht differenzierbar, so wird $t_0$ Knickstelle genannt.
\paragraph{sprungstetig:}
Eine Funktion, die bis auf endliche Sprung- und Knickstellen überall differenzierbar ist, wird sprungstetig genannt.
\paragraph{gerade:} Eine Funktion $u(t)$ ist gerade, falls ihr Graph achsensymmetrisch zur $u$-Achse ist:
\[
	u(-t) = u(t) \qquad \text{ für alle } t
\]
\paragraph{ungerade:} Eine Funktion $u(t)$ ist ungerade, falls ihr Graph punktxymmetrisch zum Ursprung ist:
\[
	u(-t) = -u(t) \qquad \text{ für alle } t
\]
\paragraph{kausale Signale:} Dies sind Funktionen, die \textit{vor} einem Zeitpunkt $t_0$ Null sind. (Bsp. der Einheitssprung)
\paragraph{beschränkt:} Ein Signal $u(t)$ heisst beschränkt, falls $u$ dem Betrage nicht beliebig grosse Werte annimmt:
\[
	\left| u(t) \right| \le M_u	\qquad \text{ für alle } t.
\]


\subsection{Operationen}